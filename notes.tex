Det som var spesielt i SR, var at det var en sammenheng mellom tid og rom. Vi tenkte på tid som en ekstra dimensjon. 

The idea of general relativity is that space time is curved, meaning that the space itself has some structure determined by the masses and energy present. So whatever is present in our universe forms the space. It is the curvature of the 4 dimentional space that determines how you move. (Imagine a ball on a frictionless bumpy road in space. The only thing determining where it moves is the 3d surface as it bumps into stuff)

I et tilstrekklig lite område kan alle effektene av krummingen sees bortifra, det vil si at man ikke har tidal forces og at tyngdekraften kun er et resultat av akselrasjon. 










Hei

Jeg har et par spøsrmål etter det vi snakket om i går.

Foreløpig er foreståelsen min av samtalen i går følgende, vær så snill å se om du fnner noen feil: 

Ekvivalensprinsippet var noe man først antok, spesielt siden man visste at alle masser faller med lik akselrasjon. Dette ga inspirasjon til å beskrive gravitasjon som en geometri som var krum globalt. Lokalt så kunne geometrien alltids beskrives som et rom der det ikke virker noen krefter, dvs ekvivalent med å være i fritt fall. 

Enhver geometri har et metrikk som beskriver hvordan man finner den korteste veien fra et punkt til et annet. Metrikken til et rom er input til geodeligningen som da gir deg den korteste veien mellom to punkter. Det vil si at metrikken gir hvordan objekter som ikke påvirkes av krefter beveger seg i rommet.

Geodeligningen for et vanlig Minkowski rom er veldig enkel da ethvert objekt vil bevege seg i en rett linje.
Siden det krummet rommet lokalt kan approksimeres som et Minkowski rom kan man gjøre en kordinattransformasjon fra det lokale Minkowski rommet
til de globale kordinatene i det krummet rommet.  Putter man denne transformaskjonen inn i geodeligningen for Minkowski rommet så får man geodeligningen i globale kordinater, det vil si geodeligningen som beskriver bevegelse i krumt rom. Denne geodeligningen inneholder metrikken til totale rommet g_{mu_nu}. Dersom man løser Einsteins feltligninger så kan man finne g_{mu_nu}.

Spørsmål:
- Jeg har fremdeles problemer med å forstå hvorfor man er avhenige av at ekvivalensprinsippet skal holde for at GR skal stemme? Ser jo at den geometriske beskrivelen tar inn at ekvivalensprinsippet holder, men er det mulig a formulere det mer konkret? 
- Blir det riktig å si at g_{mu_nu} beskriver transformasjonen mellom lokale Minkowski rom og det globale krumme rommet? 
- Er det riktig å si at Ricci tensoren beskriver ``tidal forces'', det vil si alle krefter som man ikke kan se på som en akselrasjon. 
- Til slutt et mer overordnet spørsmål: Det virker som om de aller fleste konsekvenser av GR kan utledes fra kun ekvivalensprinsippet. Hvilke konsekvenser av GR kan ikke sees utifra ekvivalensprinsippet i seg selv? 























Det vil si at man kan løse geodeligningen



Tenkte på en analogi til 3D. Dersom du har en kule så kan du alltids approksimere et flatt området i et punkt på kulen. Dersom du 

Dersom kulen består av hakk, det vil

For å kunne finne ut geodelignigen for dette krumme rommet, må man kunne gjøre approkismere  en transformasjon i ethvert punkt fra det krumme rommet til  
Om du i ethvert punkt kan anta at rommet i et lite område har kan sees på som et Minkowski rom, så kan du alltids gå 




Denne geometrien er lokalt i ethvert punkt likt Minkowski rommet, og men globalt er det krumt. (Jeg forstår ikke helt hva som menes med det. Med det må det jo menes at rommet er lokalt flatt, det vil si at det ikke virker noen krefter der. Som igjen vil si at man egentlig aldri opplever at tyngdekreftene virker på oss, men at når vi merker tyngdekreftene så er det bare at vi motsetter oss å bevege oss i rettlinjede baner i det krumme rommet.


Ved å gjøre en enkel transformasjon så kan du bli kvitt tyngdekraften, det vil si at tyngdekraften lokalt sett kun er et resultat av akselrasjon. 

I et krumt rom følger bevegelsene geodetiske kurver i følge geodeligningen.
Geodeligningen er avhenige av metrikken til rommet, det vil si en funksjon som forteller hvordan man regner ut minste avstand mellom to punkter, det vil også si den kurven som man vil følge dersom det ikke virker noen krefter.
Einsteinligningen kan brukes til å finne denne metrikken i ethvert punkt. Det vil si at vi ved Einsteiligningen kan finne det vi trenger å putte inn i geodeligningen metrikken til rommet, og da finne ut hvordan man beveger seg i rommet. 

Ekvivalensprinsippet gjør at vi kan tenke på ethvert lite området i rommet som i fritt fall. 



According to GR everyone is in contant free fall, we just resist moving in a straight line. 


Vi på jorda er egentlig alltid i fritt fall, det vil si vi beveger oss i rettlinjed i et krumt rom. Grunnen til at vi føler gravitasjonen er at vi motsetter oss de rettlinjede bevegelsene. 

When something is falling to the ground it is because whatever is falling is following the geodecids, while we are resisting it. By a simple coordinate transformation, this is equivalent to saying that we are accelerating and what is falling is standing still. It is a simple coordinate transformation.

I det lokale kordinatsystemet virker derfor SR. 

Dersom i et lite området er rommet flatt, kan du alltids skrive en transformasjon fra det lokale til det globale?

Det man tester med ekvivalensprinsipeet er om rommet der man tester kan beskrives ved et enkelt flatt rom (det vil si at man er i fritt fall)


Lokalt skal det tilsvare at man er i fritt fall, eller at man er i Minkowski rommet. Betyr ikke det at man egentlig er i et akselrerende kordinatsystem? Jeg trodde at Minkowski rommet vil si at man har konstant hastighet? Eller betyr det at det kan gjøres en enkel transformasjon fra 
